%
%  modelling.tex
% 
%  Created by Lars Yencken on 18-11-2008.
%  Copyright 2008 Lars Yencken. All rights reserved.
%

%----------------------------------------------------------------------------%
% HEADER
%----------------------------------------------------------------------------%

\documentclass[11pt,a4paper]{article}

\usepackage{url,latexsym,hyperref,graphicx,amssymb,amsmath} 
\usepackage{xspace} 
\usepackage{fullpage}
\usepackage{tweaklist}

% XeTeX specific
\usepackage{fontspec} 
\usepackage{xunicode}
\setmainfont[
  Mapping=tex-text,
%  SmallCapsFont={LMRoman10-CapsRegular},
  ItalicFont={ACaslonPro-Italic},
  BoldFont={ACaslonPro-Bold}
]{ACaslonPro-Regular}
\setmonofont{Andale Mono}

\newfontinstance\scshape[Letters=SmallCaps, Numbers=Uppercase]{ACaslonPro-Regular}

\usepackage{sectsty}
\allsectionsfont{\fontspec{ACaslonPro-Bold}}

%----------------------------------------------------------------------------%
% CJK
%----------------------------------------------------------------------------%

\newcommand{\japanesefont}{\fontspec{Epson-Kaisho}}
\newcommand{\chinesefont}{\fontspec{LiSong Pro}}

% Commands for Japanese examples
\newcommand{\ex}[1]{\textit{#1}}
\newcommand{\gloss}[1]{``#1''}
\newcommand{\jreading}[1]{[#1]}
\newcommand{\jp}[1]{\japanesefont#1\normalfont}
\newcommand{\cn}[1]{\chinesefont#1\normalfont}
\newcommand{\creading}[1]{[#1]}

%----------------------------------------------------------------------------%
% CONTENT
%----------------------------------------------------------------------------%

\title{\fontspec{ACaslonPro-Bold} Kanji Tester: Error Modelling}
\author{\href{mailto:lljy@csse.unimelb.edu.au}{Lars Yencken}}

\begin{document}
\maketitle

\tableofcontents

\section{Prior distributions}

This section documents the prior error distributions use for each error type. 

\subsection{$(\text{kanji}' | \text{kanji})$}

Given a frequency distribution $\Pr(K)$ and a similarity measure $s: K \times
K \rightarrow [0, 1]$, we can calculate the error distribution:
\[
\Pr(K'|K) = \frac{\Pr(K')s(K', K)}{\alpha}
\]
where $\alpha$ is a normalisation constant.

\subsection{$(\text{reading}' | \text{reading}, \text{kanji})$}

Describe FOKS reading model.

\section{Update rule}

This section describes how we update our error models from user responses.
When the error model is based on an individual item, we use a simple updating
method, and when it is based on a sequence of items from our error
distribution, we need to use a different sequence model.

\subsection{Simple updating}

First we start with some basic definitions:

\begin{eqnarray*}
O & = & \text{the finite set of options available for a question}\\
D & = & \text{the options displayed to the user, $D = d_1 \dots d_n$,
    $D \subset O$}\\
c & = & \text{the option the user picks}\\
\end{eqnarray*}

We displayed a random subset $D$ of possible options $O$ to the user, and they
chose option $c$ as their answer. $\Pr(c|D)$ is known from our prior
distribution. We wish to determine the posterior value for $\Pr'(c)$.
We base our update rule on the constraint:

\[
\forall_{\{i:d_i \ne c\}} \Pr(c|D) \ge \Pr(d_i|D) + \epsilon
\]

That is, the user chose $c$ because it was better than any other option by a
margin of $\epsilon$. Our update rule simply enforces this margin of
$\epsilon$ in the posterior distribution $(C|D)$ used in the next iteration.

\begin{enumerate}
  \item Let $m = \max_{\{i: d_i \ne c\}} \Pr(d_i|D) + \epsilon$
  \item Define the posterior distribution $(C|D)$ as follows:
  \begin{itemize}
    \item $\Pr'(d_i|D) = \Pr(d_i|D)$ if $d_i \ne c$
    \item $\Pr'(c|D) = \max\{\Pr(c|D), m\}$
    \item Normalise $(C|D)$ such that $\sum_i \Pr'(d_i|D) = 1$
  \end{itemize}
  \item Retain $\Pr'(d) = \Pr(d)$ for all $d \in O \setminus D$\footnote{The evidences gives us no reason to
    increase or decrease the overall likelihood of $D$.}
  \item Set $\Pr'(d_i) = \Pr'(d_i|D)\Pr'(D)$
\end{enumerate}

\subsection{Updating sequence models}

\subsubsection{Draft model}
In our sequence model, each distractor $d$ is a sequence of elementary items
$d_1 \dots d_n$ in the distractor space $D = D_1 \times \dots \times
D_n$.\footnote{Note the differing meaning of this subscript from the previous
section. $d_i$ previously represented the $i$th option displayed to the user;
it now represents the $i$th sequence element of the single option $d$.}
We use the same steps as in the simple update, up to the point where we have
calculated $\forall d : \Pr'(d|D)$. At this point, we know the new sequence
probability, but wish to determine the new primitive probability
$\Pr'(d_i|D_i)$. We define:
\[
\Delta = \frac{{\Pr}'(d|D)}{\Pr(d|D)}
\]
which has known value at this stage. We make some simplifying assumptions to
further reduce the right-hand side:
\begin{eqnarray*}
\Pr(d|D) & = & \Pr(d_{1} \dots d_{n}|D)\\
& \approx & \prod_{j=1}^n \Pr(d_j|D)\\
& & \text{assuming indepencence of each $(d_{j}|D)$}\\
& \approx & \prod_{j=1}^n \Pr(d_j|D_j)\\
& & \text{assuming further independence of $d_j$ from all $D_i$, $i \ne j$}
\end{eqnarray*}
Whilst these assumptions clearly {\it do not} hold in real life, whether our
sequences models are of kanji readings or of highly similar characters, since 
real-life cooccurrence frequencies differ significantly for different
combinations of sequence items. We nonetheless make these assumptions to retain a simple and tractable model. Plugging this result back into our initial recipe for $\Delta$:
\[
\Delta \approx \prod_{j=1}^n \frac{{\Pr}'(d_j|D_j)}{\Pr(d_j|D_j)}
\]
We choose to distribute the probability mass evenly, which adds the
constraint:
\[
\frac{{\Pr}'(d_j|D_j)}{\Pr(d_j|D_j)} = \frac{{\Pr}'(d_k|D_k)}{\Pr(d_k|D_k)}
\mbox{\hspace{0.5cm}}
\forall j, k \in \{{1, \dots, n}\} 
\]
Then our final update rule emerges:
\[
{\Pr}'(d_j|D_j) = \Delta^{1/n} \Pr(d_j|D_j)
\]

Note that this rule requires us to change our earlier constraint that no unseen
options should have their likelihoods changed. At the primitive level, any
unseen symbol will retain its original likelihood. However, at the sequence
level, unseen combinations of seen primitives {\it will} have their
likelihoods changed as the likelihoods of their constitutents are updated.
This is not problematic, merely a consequence of our unigram model.

\appendix

\newpage
\section{Alternative error models}

\subsection{Linear interpolation}

\subsubsection{Description}

An alternative error model which could be considered is one where fixed priors
are used to model different sources of error, and linear interpolation these
alternating error models are used. We use an example to illustrate, that where
we want to know the probability of the distractor \jp{にっき} for word
\jp{日本}. Let:
\begin{eqnarray*}
R & = & R_1 R_2 \text{\hspace{0.5cm}the sequence reading}\\
& = & \text{\jp{にっき}}\\
R_1 & = & \text{\jp{にっ}\hspace{0.5cm}the segment reading}\\
R_1^{*} & = & \text{\jp{にち}\hspace{0.5cm}the canonical segment reading}\\
R_2 = R_2^{*} & = & \text{\jp{き}}\\
K & = & K_1 K_2\\
& = & \text{\jp{日本}}\\
K_2' & = & \text{\jp{木}\hspace{0.5cm}a similarity variant of $K_2$}
\end{eqnarray*}
We construct our probability of seeing this sequence reading as follows: 
\begin{eqnarray*}
\Pr(R|K) & = & \alpha_\text{rc}\prod_{i = 1}^n
{\Pr}_{\text{rc}}(R_i^{*}|K_i) + \\
& & \beta_\text{vl}\prod_{i = 1}^n {\Pr}_\text{vl}(R_i|R_i^{*}) + \\
& & \gamma_\text{on}\prod_{i = 1}^{n-1} {\Pr}_\text{on}(R_i|R_i^{*})
+\\
& & \delta_\text{sv}\prod_{i = 2}^{n} {\Pr}_\text{sv}(R_i|R_i^{*}) +
\\
& & \epsilon_\text{gs}\prod_{i = 1}^n
{\Pr}_\text{gs}(R_i^{*}|K_i'){\Pr}_\text{gs}(K_i'|K_i)
\end{eqnarray*}
This model uses many fixed probability tables for reading confusion (rc),
vowel length (vl), sound euphony (on), sequential voicing (sv) and graphical
similarity (gs). The final probability is determined by using the user-specific
vector
\[
\langle \alpha_\text{rc}, \beta_\text{vl}, \gamma_\text{on}, \delta_\text{sv},
\epsilon_\text{gs} \rangle
\]
where each value is the number of times a user has made a mistake generated
using that model. After the user makes a number of responses, the vector is
updated with the new error counts. A normalised version of the updated vector
is then used to generate the next set of test questions.

\subsubsection{Comparison}

Comparing this model to our chosen model yields advantages, disadvantages, and
differences which could be either.

\begin{center}
\begin{tabular}{p{4.5cm}p{4.5cm}p{4.5cm}}
\hline
{\bf Advantages} & {\bf Disadvantages} & {\bf Differences}\\
\hline
\begin{list}{\labelitemi}{\leftmargin=1em \itemindent=-0.3em \topsep=0em}
  \item Simple per-user model
  \item Simple update function
  \item Easy interpretation as error type distribution
\end{list}
&
\begin{list}{\labelitemi}{\leftmargin=1em \itemindent=-0.3em \topsep=0em}
  \item Monolithic error model more complex
  \item Explicit error distributions must be reconstructed
\end{list}
&
\begin{list}{\labelitemi}{\leftmargin=1em \itemindent=-0.3em \topsep=0em}
  \item Adapts to the type of error a learner is making, not the exact error
  itself
\end{list}
\\
\hline
\end{tabular}
\end{center}

The two error models consider learner error at a different level of
abstraction, but ultimately neither limits the forms of post-hoc evaluation
which can be performed. Ideally, both would be implemented and compared on
actual user data, to see which more accurately predicts user responses. 

\end{document}
